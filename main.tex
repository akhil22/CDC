% Copyright 2004 by Till Tantau <tantau@users.sourceforge.net>.
%
% In principle, this file can be redistributed and/or modified under
% the terms of the GNU Public License, version 2.
%
% However, this file is supposed to be a template to be modified
% for your own needs. For this reason, if you use this file as a
% template and not specifically distribute it as part of a another
% package/program, I grant the extra permission to freely copy and
% modify this file as you see fit and even to delete this copyright
% notice. 

\documentclass{beamer}
\usepackage{graphics} % for pdf, bitmapped graphics files
\usepackage{graphicx}
\usepackage{verbatim} 
%\usepackage{pstricks}
%\usepackage{epstopdf}
% Generated with LaTeXDraw 2.0.8
% Wed Feb 18 16:25:36 IST 2015
\usepackage{amsmath} % assumes amsmath package installed
%\usepackage{amssymb}  % assumes amsmath package installed
\usepackage{algorithm} % assumes amsmath package installed
\usepackage{algpseudocode}
\usepackage{subfigure}
\usepackage{multirow}
\usepackage{algpseudocode}
\usepackage[pdf]{pstricks}
\usepackage{epsfig}
\usepackage{pst-grad} % For gradients
\usepackage{pst-plot} % For axes
\usepackage{epsfig} % for post
% There are many different themes available for Beamer. A comprehensive
% list with examples is given here:
% http://deic.uab.es/~iblanes/beamer_gallery/index_by_theme.html
% You can uncomment the themes below if you would like to use a different
% one:
%\usetheme{AnnArbor}
%\usetheme{Antibes}
%\usetheme{Bergen}
%\usetheme{Berkeley}
%\usetheme{Berlin}
%\usetheme{Boadilla}
\usetheme{boxes}
%\usetheme{CambridgeUS}
%\usetheme{Copenhagen}
%\usetheme{Darmstadt}
%\usetheme{default}
%\usetheme{Frankfurt}
%\usetheme{Goettingen}
%\usetheme{Hannover}
%\usetheme{Ilmenau}
%\usetheme{JuanLesPins}
%\usetheme{Luebeck}
%\usetheme{Madrid}
%\usetheme{Malmoe}
%\usetheme{Marburg}
%\usetheme{Montpellier}
%\usetheme{PaloAlto}
%\usetheme{Pittsburgh}
%\usetheme{Rochester}
%\usetheme{Singapore}
%\usetheme{Szeged}
%\usetheme{Warsaw}

\title{Mobile Robot Navigation Amidst Humans with Intents and Uncertainties:A Time Scaled Collision cone Approach}

% A subtitle is optional and this may be deleted
\subtitle{}

\author{Akhil Nagariya\inst{1} \and Bharath Gopalakrishna\inst{1} \and Arun Singh\inst{2} \and Krishnam Gupta \inst{1} \and K Madhava Krishna \inst{1}}
% - Give the names in the same order as the appear in the paper.
% - Use the \inst{?} command only if the authors have different
%   affiliation.

\institute[Universities of Somewhere and Elsewhere] % (optional, but mostly needed)
{
  \inst{1}%
  RRC\\
  IIIT Hyderabad
  \and
  \inst{2}%
  Ben-Gurion University, Isreal
  }
% - Use the \inst command only if there are several affiliations.
% - Keep it simple, no one is interested in your street address.

\date{CDC 2015}
% - Either use conference name or its abbreviation.
% - Not really informative to the audience, more for people (including
%   yourself) who are reading the slides online

\subject{Theoretical Computer Science}
% This is only inserted into the PDF information catalog. Can be left
% out. 

% If you have a file called "university-logo-filename.xxx", where xxx
% is a graphic format that can be processed by latex or pdflatex,
% resp., then you can add a logo as follows:

% \pgfdeclareimage[height=0.5cm]{university-logo}{university-logo-filename}
% \logo{\pgfuseimage{university-logo}}

% Delete this, if you do not want the table of contents to pop up at
% the beginning of each subsection:
\AtBeginSubsection[]
{
  \begin{frame}<beamer>{Outline}
    \tableofcontents[currentsection,currentsubsection]
  \end{frame}
}

% Let's get started
\begin{document}

\begin{frame}
  \titlepage
\end{frame}

\begin{frame}{Outline}
  \tableofcontents
  % You might wish to add the option [pausesections]
\end{frame}

% Section and subsections will appear in the presentation overview
% and table of contents.
\section{Motivation}


\subsection{}

\begin{frame}{Motivation}
  \begin{itemize}
  \item {
   Robots and humans are beginning to occupy the same work spaces
  }
  \item {
   Account for human intent in robot's navigation and avoidance Maneuver
  }
  \item {
    Uncertain and Haphazard local movements of human 
  }
  \end{itemize}
\end{frame}


%\subsection{Second Subsection}

% You can reveal the parts of a slide one at a time
% with the \pause command:
%\begin{frame}{Second Slide Title}
 % \begin{itemize}
 % \item {
  %  First item.
   % \pause % The slide will pause after showing the first item
 % }
 % \item {   
  %  Second item.
  %}
  % You can also specify when the content should appear
  % by using <n->:
  %\item<3-> {
   % Third item.
  %}
  %\item<4-> {
   % Fourth item.
  %}
  % or you can use the \uncover command to reveal general
  % content (not just \items):
  %\item<5-> {
   % Fifth item. \uncover<6->{Extra text in the fifth item.}
  %}
  %\end{itemize}
%\end{frame}

\section{Human Intention prediction}

\subsection{}

\begin{frame}{Human Intention prediction}
%\begin{block}{Block Title}
%You can also highlight sections of your presentation in a block, with it's own title
%\end{block}
%\begin{theorem}
%There are separate environments for theorems, examples, definitions and proofs.
%\end{theorem}
%\begin{example}
%Here is an example of an example block.
%\end{example}
\begin{itemize}
\item{Characterize intents as the final destinations a person might reach}
\item{Let $D = \{\mathbf{d^1,d^2,...,d^m}\}$ be the set of final destinations a person can go to in a given environment}
\item{compute the probability of each of these intents Using Hidden Markov Model.}
\item{Characterize local Haphazard movements as a gaussian $\mathcal{N}(\mu_i(\mathbf{x}^{t}),\sigma_t)$ }
\end{itemize}
\end{frame}
\begin{frame}{Human Intention prediction}
\begin{figure}
\centering
\includegraphics[width= 8.1cm, height=4.5cm]{screen1.eps}
\end{figure}
\end{frame}

% Placing a * after \section means it will not show in the
% outline or table of contents.
\section{collision avoidance}
\section*{Summary}

\begin{frame}{Summary}
  \begin{itemize}
  \item
    The \alert{first main message} of your talk in one or two lines.
  \item
    The \alert{second main message} of your talk in one or two lines.
  \item
    Perhaps a \alert{third message}, but not more than that.
  \end{itemize}
  
  \begin{itemize}
  \item
    Outlook
    \begin{itemize}
    \item
      Something you haven't solved.
    \item
      Something else you haven't solved.
    \end{itemize}
  \end{itemize}
\end{frame}



% All of the following is optional and typically not needed. 
\appendix
\section<presentation>*{\appendixname}
\subsection<presentation>*{For Further Reading}

\begin{frame}[allowframebreaks]
  \frametitle<presentation>{For Further Reading}
    
  \begin{thebibliography}{10}
    
  \beamertemplatebookbibitems
  % Start with overview books.

  \bibitem{Author1990}
    A.~Author.
    \newblock {\em Handbook of Everything}.
    \newblock Some Press, 1990.
 
    
  \beamertemplatearticlebibitems
  % Followed by interesting articles. Keep the list short. 

  \bibitem{Someone2000}
    S.~Someone.
    \newblock On this and that.
    \newblock {\em Journal of This and That}, 2(1):50--100,
    2000.
  \end{thebibliography}
\end{frame}

\end{document}


